	\newpage
\section{Projektowanie}		%3
%Opis przygotowania narzędzi (git, visual studio). Wybór i opis bibliotek, klas. Szkic layoutów. Pseudo kody. Opisy wykorzystanych algorytmów (np. algorytm sortowania). Dokładniejsze określenie założeń i działania aplikacji, (np.: ten przycisk otworzy takie okno a w tym oknie wpisujemy takie dane).

\subsection{Przygotowanie narzędzi Git oraz Visual Studio}

\hspace{1cm}W celu korzystania z narzędzia Git należy utworzyć na swoim komputerze repozytorium lokalne oraz dodać do niego pliki projektu. Na platformie GitHub utworzyć repozytorium zdalne, które umożliwi wszystkim autorom projektu współprace przy tworzeniu aplikacji.
\newline
W Visual Studio należy zainstalować dodatek opracowywanie aplikacji mobilnych za pomocą środowiska Xamarin oraz zestaw Android SDK, który umożliwia korzystanie z emulatora.
Do tworzenia projeku wybraliśmy szablo aplikacji mobilnej Xamarin.Forms.

\subsection{Biblioteki i klasy}
\hspace{1cm} Biblioteka Xamarin.Essentials udostępnia międzyplatformowy interfejs programowania aplikacji mobilnych (API). Jest ona dostępna jako pakiet NuGet i jest uwzględniana w każdym nowym projekcie w programie Visual Studio.
Biblioteka ta oferuje klasy, które zostaną przez nas użyte podczas tworzenia projektu.
\newline
Geolokalizacja:
\newline
Klasa Geolocation udostępnia interfejsy API do pobierania bieżących współrzędnych geolokalizacji urządzenia.
\newline
Motyw aplikacji:
\newline
Interfejs API RequestedTheme jest częścią AppInfo klasy i zawiera informacje dotyczące tematu żądanego dla uruchomionej aplikacji przez system.
\newline
Akcelerometr:
\newline
Klasa Accelerometer umożliwia monitorowanie czujnika przyspieszeniomierza urządzenia, który wskazuje przyspieszenie urządzenia w trójwymiarowej przestrzeni.
\newline
Klasa GoogleMap:
\newline
Firma Google oferuje natywny interfejs API mapowania dla systemu Android. Pozwala on na zmienianie punktu widzenia mapy, dodawanie i dostosowywanie znaczników, oznaczanie mapy za pomocą nakładek.
\newline
Wymagania wstępne Mapy API usługi Google: uzyskanie klucza Mapy API, zainstalowanie pakietu Xamarin.GooglePlayServices i Mapy pakietu z NuGet, określenie wymaganych uprawnień.
\newline
Klasa GoogleMap poprzez aplikację platformy Xamarin.Android będzie współdziałała z aplikacją Google Maps.
\newline
\newline
Biblioteka Microsoft Authentication Library (MSAL) pozwala na dodawanie uwierzytelniania do aplikacja. Umożliwi to logowanie do aplikacji przy użyciu Google lub Facebook.
\newline
Klasa WebAuthenticator umożliwia inicjowanie przepływów opartych na przeglądarce, które nasłuchują wywołania zwrotnego do określonego adresu URL zarejestrowanego w aplikacji.
\newline



 
 