	\newpage
\section{Implementacja}		%4
%Wkleić szkielet kodu, wraz z komentarzami. Opisać zmienne, struktury do czego służą. Opisać procedury, metody co wykonują. Opisać nowe zdefiniowane klasy. Opisać dziedziczenie. Opisać nowo utworzone pliki za co odpowiadają.
Tworznie Menu: \newline
W \textbf{activity\_main\_drawer.xml}:
\newline
- \textbf{andriod:id} - nadanie nazwy elementu
\newline
- \textbf{android:icon} - wybranie jednej z domyślnych ikon do menu
\newline
- \textbf{android:title} - dodaje napis, który będzie wyświetlany na elemencie
\newline
Przykład:
\newline  
\textbf{android:id="@+id/nav\_mapmode" \newline
android:icon="@android:drawable/ic menu\_mapmode" \newline
android:title="Mapa"} \newline
W \textbf{MainActivity.cs}:
\newline
W funkcji \textbf{public boolOnNavigationItemSelected(IMenuItem item)} warunek \textbf{id == Resource.ID.[nazwa elementu]} określa dla którego elementu będzie wykonywana dana instrukcja. \newline
Przykład: \newline
\textbf{if (id == Resource.Id.nav\_mapmode) \newline
\{ \newline
  \newline
\}}
 


